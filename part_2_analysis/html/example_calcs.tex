
% This LaTeX was auto-generated from MATLAB code.
% To make changes, update the MATLAB code and republish this document.

\documentclass{article}
\usepackage{graphicx}
\usepackage{color}

\sloppy
\definecolor{lightgray}{gray}{0.5}
\setlength{\parindent}{0pt}

\begin{document}

    
    
\subsection*{Contents}

\begin{itemize}
\setlength{\itemsep}{-1ex}
   \item Temperature Calculation
\end{itemize}
\begin{verbatim}
clear all;
close all;
clc;
\end{verbatim}


\subsection*{Temperature Calculation}

\begin{par}
DESCRIPTIVE TEXT
\end{par} \vspace{1em}
\begin{par}
Conversion from a Type T thermocouple voltage $V$ to a temperature T is done by applying a set of coefficients to the recorded voltage. The conversion coefficients may be seen below.
\end{par} \vspace{1em}
\begin{verbatim}
T0 = 1.3500000E+02;
V0 = 5.9588600E+00;
p1 = 2.0325591E+01;
p2 = 3.3013079E+00;
p3 = 1.2638462E-01;
p4 = -8.2883695E-04;
q1 = 1.7595577E-01;
q2 = 7.9740521E-03;
q3 = 0.0;

% EQUATION GOES HERE

% Where $$\Delta{V} = V - v_o$$

% The calculation done in code(with $V$ in millivolts is written as:

V =  1.4094;

delta_v = V - V0;

numerator = delta_v.*( p1 + delta_v.*( p2 + delta_v.*(p3 + p4.*delta_v)));
denominator = 1 + delta_v.*(q1 + delta_v.*(q2 + q3.*delta_v));
T = T0 + numerator./denominator;

% Now printing the result:

T
\end{verbatim}

        \color{lightgray} \begin{verbatim}
T =

   35.1559

\end{verbatim} \color{black}
    


\end{document}
    
